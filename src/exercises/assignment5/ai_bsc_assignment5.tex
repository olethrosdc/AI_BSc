\documentclass[11pt]{article}
\usepackage{parskip}

\usepackage{../assets/tp}
\usepackage[french]{babel}
\usepackage[utf8]{inputenc}
\usepackage{tgcursor}
\usepackage[T1]{fontenc}
\usepackage[]{../assets/subfigure}
\usepackage{xspace}
\usepackage{url}
\usepackage{ifthen}
\usepackage{amsmath}
\usepackage{amssymb}
\usepackage[usenames,dvipsnames]{xcolor}
\usepackage{hyperref}
\usepackage{listings}
\usepackage{dirtree}
\usepackage[scaled]{helvet}
\usepackage{gensymb}
\usepackage{epsfig}
\usepackage{tikz}
\usepackage{centernot}
\usepackage{verbatim}
\usetikzlibrary{arrows,automata, shapes, backgrounds, positioning}

\renewcommand{\thesubsection}{\arabic{subsection}}
\makeatletter
\def\@seccntformat#1{\@ifundefined{#1@cntformat}%
   {\csname the#1\endcsname\quad}%    default
   {\csname #1@cntformat\endcsname}}% enable individual control
\newcommand\section@cntformat{}     % section level 
\makeatother

\lstset{
     extendedchars=true,
     literate=%
         {á}{{\'a}}1
         {í}{{\'i}}1
         {é}{{\'e}}1
         {è}{{\`e}}1
         {ý}{{\'y}}1
         {ú}{{\'u}}1
         {ó}{{\'o}}1
         {ě}{{\v{e}}}1
         {š}{{\v{s}}}1
         {č}{{\v{c}}}1
         {ř}{{\v{r}}}1
         {ž}{{\v{z}}}1
         {ď}{{\v{d}}}1
         {ť}{{\v{t}}}1
         {ň}{{\v{n}}}1
         {ů}{{\r{u}}}1
         {Á}{{\'A}}1
         {Í}{{\'I}}1
         {É}{{\'E}}1
         {Ý}{{\'Y}}1
         {Ú}{{\'U}}1
         {Ó}{{\'O}}1
         {Ě}{{\v{E}}}1
         {Š}{{\v{S}}}1
         {Č}{{\v{C}}}1
         {Ř}{{\v{R}}}1
         {Ž}{{\v{Z}}}1
         {Ď}{{\v{D}}}1
         {Ť}{{\v{T}}}1
         {Ň}{{\v{N}}}1
         {Ů}{{\r{U}}}1
}

\hypersetup{
  colorlinks=true,
  linkcolor=blue!50!red,
  urlcolor=green!70!black
}
\numberwithin{equation}{section}

\newcommand{\sol}[1]{
    %{\color{blue} #1}
    }
\renewcommand{\deg}[1]{\:^\circ\text{#1}}

\renewcommand{\familydefault}{\sfdefault}

\renewcommand*\DTstylecomment{\rmfamily}

\definecolor{codegray}{rgb}{0.0,0.0,0.7}
\lstset{float,
  commentstyle=\color{codegray},
  basicstyle=\scriptsize,
  columns=fullflexible,
  numbers=left,
  numberstyle=\tiny,
  numbersep=5pt,
  xleftmargin=5pt,
  frame=tb,
  captionpos=b,
  mathescape=true,
  tabsize=2,
  showtabs=true,
  escapeinside={(*@}{@*)}}


\newboolean{showcomments}
\setboolean{showcomments}{true}
\ifthenelse{\boolean{showcomments}}
{ \newcommand{\mynote}[3]{
    \fbox{\bfseries\sffamily\scriptsize#1}
    {\small$\blacktriangleright$\textsf{\emph{\color{#3}{#2}}}$\blacktriangleleft$}}}
{ \newcommand{\mynote}[3]{}}
\newcommand{\shrink}[1]{}
% One command per author:
\newcommand{\pf}[1]{\mynote{Pascal}{#1}{magenta}}
\newcommand{\cg}[1]{\mynote{Christian}{#1}{orange}}
\newcommand{\rd}[1]{\mynote{R\'emi}{#1}{purple}}
\newcommand{\answer}[1]{\mynote{Answer:}{#1}{blue}}

\makeatletter
    \def\blfootnote{\xdef\@thefnmark{}\@footnotetext}
\makeatother

%%%%%%%%%%%%%%%%%%%%%%%%%%%%%%%%

\quand{2023-2024}
\siglemat{Artificial Intelligence}
\sorte{Week}
\formation{Computer science department}
\titre{Assignment 5}

\usepackage{courier}

\begin{document}
\thispagestyle{plain}
\mettreletitre

\blfootnote{Questions ? victor.villin@unine.ch}

\section*{Subject}

Topics of this session:
\begin{enumerate}
\item Conditional probability and independence.
\item Bayes’s theorem.
\item Belief Networks.
\end{enumerate}
\textbf{This assignement is graded and must be submitted (individually) on Moodle before next week's class}. 

\smallskip
For each exercise, detail your reflexion steps:
\begin{itemize}
    \item We are mostly interested in your actual thinking process.
    \item Even if you are unable to solve an exercise, write out what were you reflexion steps.
    \item For each attempted exercise, a written feedback will be provided (if time alllows it). 
\end{itemize}

For coding:
\begin{itemize}
    \item \href{https://noto.epfl.ch}{Noto} (Online Jupyter NoteBook).
    \item Any other python coding environment you prefer using.
\end{itemize}

\section*{Exercise 1}
Let's consider the $N$-meteorologist problem. Assume there are $N=3$ weather stations, which predict what is the chance that it's going to rain or not.
In the beginning we believe equally each weather station forecasts.

\textbf{Tasks:}
\begin{enumerate}
    \item Represent our belief network, where nodes are the events, and directed edges represent relationships between events.

    \sol{
        \paragraph{Notations}
        \begin{itemize}
            \item $C_i$: Station $i$ predicts correctly the weather.
            \item $R$: It will rain.
        \end{itemize}
        Using those notations, we can draw this belief network representing the relationships between variables:
        
        {\centering
        \begin{tikzpicture}[node distance=1cm and 2cm, every node/.style={minimum width=1cm}, every path/.style={draw, thick}, edge/.style = {->,> = latex', >=stealth}]
            \node (C2) [draw, circle] {$C_2$};
            \node (C1) [draw, circle, left=of C2] {$C_1$};
            \node (C3) [draw, circle, right=of C2] {$C_3$};
            \node (R) [draw, circle, above=of C2] {$R$};
            
            % Edges
            \draw[edge] (R) to (C1);
            \draw[edge] (R) to (C2);
            \draw[edge] (R) to (C3);
        \end{tikzpicture}
        }
    }

    \item Express the probabilities of rain for the first day, using our initial belief, based on the following forecasts:
    \begin{table}[h]
        \centering
        \begin{tabular}{ c | c c c }
          & Station 1 & Station 2 & Station 3 \\ 
         \hline
         Rain Probability & 0.5 & 0.7 & 0.05 \\
        \end{tabular}
        \caption{Forecasts for the first day.}
    \end{table}

    \sol{
        We have an initial belief that we should believe equally in all three stations, that is: $P(C_i) = 1/3$, for $i\in\{1,2,3\}$.
        Additionally, the entries in the table correspond to the rain probabilities given each station $i$'s forecast: $P(R | C_i)$.
        Thus:

        \begin{equation*}
            P(R) = \sum_{i=1}^3 P(R|C_i) P(C_i) = \dfrac{0.5 \times 0.7 \times 0.05}{3} \approx 0.417.
        \end{equation*}
        With our initial belief, we believe that it will rain with probability 0.417.
    }
    \item Using Baye's Theorem, update our belief about stations being correct, knowing that it \emph{rained} on the first day.


    \sol{

        Let's update our belief for all three stations. For each station $i$, we have:
        \begin{equation*}
            P(C_i | R) = \dfrac{P(R | C_i) P(C_i)}{P(R)}.
        \end{equation*}

        We end up with:
        \begin{itemize}
            \item $P(C_i | R) = 0.4$,
            \item $P(C_i | R) = 0.56$,
            \item $P(C_i | R) = 0.04$.
        \end{itemize}

        What happens is that we believe a lot more more in the second station now, but almost no longer believe in the third one.

    }

    \item Assuming that the weather is independent of the past, update our belief about stations being correct, knowing that it also \emph{rained} on the second day, with the following forecasts:
    \begin{table}[h]
        \centering
        \begin{tabular}{ c | c c c }
            & Station 1 & Station 2 & Station 3 \\ 
           \hline
           Rain Probability & 0.3 & 0.2 & 0.2 \\
          \end{tabular}
        \caption{Forecasts for the second day.}
    \end{table}

    \sol{

    \paragraph{Additional notations}
    \begin{itemize}
        \item $C^2_i$: station $i$'s forecast is correct for the second day.
        \item $R^2$: it will rain on the second day.
    \end{itemize}

    Similarly to before, we first compute the probability of rain using our updated belief:

    \begin{equation*}
        P(R^2|R) = P(R^2) = \sum_{i=1}^3 P(R^2|C_i, R) P(C_i | R) = 0.3 \times 0.4 + 0.2 \times 0.56 + 0.2 \times 0.04 = 0.24.
    \end{equation*}

    We then compute the posteriors knowing that it rained on the second day:
    \begin{equation*}
        P(C_i | R^2, R) = \dfrac{P(R^2 | C_i, R) P(C_i | R)}{P(R^2 | R)} = \dfrac{P(R^2 | C_i) P(C_i | R)}{P(R^2)}.
    \end{equation*}

    \begin{itemize}
        \item $P(C_i | R^2, R) = 0.5$,
        \item $P(C_i | R^2, R) \approx 0.467$,
        \item $P(C_i | R^2, R) \approx 0.033$.
    \end{itemize}
    }

\end{enumerate}

\section*{Exercise 2}
We are given a coin, but we do not know its bias toward heads or tails. Instead, we model our belief about the probability of heads, denoted as , using a Beta distribution:

\begin{equation*}
    \theta \sim \text{Beta}(\alpha, \beta)
\end{equation*}

where $\theta$ represents the probability of heads, and $\alpha$ and $\beta$ are the parameters representing our belief.

Reminders:
\begin{itemize}
    \item The posterior distribution for the Beta Distribution remains a Beta distribution with updated parameters:
\[
\theta \mid \text{data} \sim \text{Beta}(\alpha + k, \beta + (n-k)).
\]
where in our case, $k$ would be the number of heads and $n-k$ the number of tails observed in the data.
    \item The mean of a Beta distribution $\text{Beta}(\alpha, \beta)$ is given by:
    \[
    E[\theta] =  \dfrac{\alpha}{\alpha + \beta}.
\]
\end{itemize}

\textbf{Tasks:}
\begin{enumerate}
\item Assume we start with a uniform prior, meaning we have no prior knowledge about the coin's bias. What values should we set for $\alpha$ and $\beta$ ?

\sol{
    Since we have no prior knowledge, we should initialise our belief with parameters that do not favor tail or head probabilities: $\alpha = \beta$.
}

\item Assuming coin tosses are independent, if we initially set \( \alpha = 2 \) and \( \beta = 2 \), and after 10 coin flips we observe 7 heads, compute the posterior distribution of \( \theta \).

\sol{
    Denoting $D$ the 10 observed tosses, we have $\theta | D \sim \text{Beta}(9, 5)$. 
}

\item Compute the mean of the posterior distribution and interpret its meaning in the context of estimating the coin's bias.

\sol{
    The mean of the posterior is as follows: $E[\theta | D] = \dfrac{9}{9 + 5} \approx 0.643$.
     Under this posterior belief, we believe that the coin is slightly biased, such that we expect to get heads 64.3\% of the time.
}
\item Implement \texttt{BetaConjugatePrior} in \texttt{coin\_toss.ipynb} and observe how the belief changes as observations come. What can you say about the model after many observations ?
\end{enumerate}


\section*{\textbf{Project -- Part 3}}

\textbf{This whole section must be done in groups of \underline{2-3 people}.}
\smallskip

This week, we will assume that some observations are stochastic. Our goal is to perform \textbf{Bayesian inference} to update our belief about the true model of the environment.

\subsection{Guided Project}

\paragraph{Dungeon Gridworld} We will once more update the assumptions about the world, adding randomness in observations.

\begin{figure}[h]
    \centering
\begin{tikzpicture}[scale=1]

    % Draw outer walls (5x5 grid with walls around it)
    \draw[thick] (-1, -1) rectangle (6, 6);

    % Draw internal grid
    \foreach \x in {0,...,5} {
        \foreach \y in {0,...,5} {
            \draw (\x, \y) rectangle ++(1, 1);
        }
    }
    
    % Draw surrounding walls as filled gray cells
    \foreach \pos in {
        (-1,-1), (0,-1), (1,-1), (2,-1), (3,-1), (4,-1), (5,-1), (5,-1),
        (-1,2), (5,2),
        (-1,3), (5,3),
        (-1,4), (5,4),
        (-1,5), (5,5),
        (-1,5), (0,5), (1,5), (2,5), (3,5), (4,5), (5,5), (5,5),
        (-1, 1), (-1, 0), (5, 1), (5, 0),
        % walls
        (2, 3), (2, 2), (0, 2), (2, 0)
    } {
        \fill[black] \pos rectangle ++(1, 1);
    }

    % Start position
    \node at (0.5, 0.5) {\includegraphics[width=0.5cm]{happy-man.png}};

    % Holes
    \node at (0.5, 4.5) {\includegraphics[width=0.5cm]{hole.png}};
    \node at (3.5, 2.5) {\includegraphics[width=0.5cm]{hole.png}};


    % Stairs
    \node at (3.5, 4.5) {\includegraphics[width=0.5cm]{stairs.png}};


    % Add small coordinate labels at the bottom right
    \foreach \x in {0,...,5} {
        \foreach \y in {0,...,5} {
            \node[scale=0.5, anchor=south east] at (\x + 0.98, \y + 0.01) {\small (\x, \y)};
        }
    }

    % sensing
    %\draw[red, line width=1.5pt] (-0.99, -0.99) rectangle (1.99, 1.99);
    %\node[anchor=south, scale=0.75] at (0.01, 2) {\color{red} Sensing Range};

\end{tikzpicture}
\label{fig:env}
\caption{A simple dungeon room with holes. The agent starts at position (0, 0).}
\end{figure}

\textbf{New Assumptions:}
\begin{itemize}
    \item {\color{gray} Falling into a hole causes the game to end immediately.}
    \item The goal is to reach the stairs (the number of steps we take does not matter this time). 
    \item The agent \textbf{has partial and noisy observations}:
    \begin{itemize}
        \item[-]  Whenever the agent is adjacent to a hole, it hears an \emph{Echo} from it with probability $p_E = \dfrac{1}{3}$.
        \item[-]  We assume we can observe perfectly our position.
        \item[-]  For simplicy, we assume we do not sense anything else (no \emph{Bump} and \emph{Light}).
        \item[-]  We have an initial belief that all tiles have a probability 0.1 to be a hole.
    \end{itemize}
\end{itemize}

\textbf{Tasks:}
\begin{enumerate}
    \item In \texttt{dungeon\_inference.ipynb}, implement the logic for updating the belief state.
    \item Ensure that the agent properly learns about the presence of holes for each tile. \textbf{Note:} Since the agent \textbf{does} not observe whether it falls into a hole, it \textbf{cannot} update its belief using that information.
    \item (Optional) Try improving the agent (implement \texttt{smarter\_agent()}) with a simple rule, such as: "\textit{If I believe that a tile has no hole with probability at least 0.9, I can go there; otherwise, I avoid it.}".
\end{enumerate}

\subsection{Personal Project}

For your project, follow these steps:
\begin{enumerate}
\item Introduce \textbf{randomness} to certain observations and implement the corresponding changes in your environment.
\item Code a simple \textbf{random agent} that selects actions uniformly at random.
\item Define an initial belief for your agent about the environment.
\item Implement a simple \textbf{Bayesian inference method} to update the agent's belief about the environment, using data collected by the random agent.
\end{enumerate}

\end{document}

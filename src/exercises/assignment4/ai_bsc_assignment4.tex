\documentclass[11pt]{article}
\usepackage{parskip}

\usepackage{../assets/tp}
\usepackage[french]{babel}
\usepackage[utf8]{inputenc}
\usepackage{tgcursor}
\usepackage[T1]{fontenc}
\usepackage[]{../assets/subfigure}
\usepackage{xspace}
\usepackage{url}
\usepackage{ifthen}
\usepackage{amsmath}
\usepackage{amssymb}
\usepackage[usenames,dvipsnames]{xcolor}
\usepackage{hyperref}
\usepackage{listings}
\usepackage{dirtree}
\usepackage[scaled]{helvet}
\usepackage{gensymb}
\usepackage{epsfig}
\usepackage{tikz}
\usepackage{centernot}
\usepackage{verbatim}
\usetikzlibrary{arrows,automata, shapes, backgrounds}

\renewcommand{\thesubsection}{\arabic{subsection}}
\makeatletter
\def\@seccntformat#1{\@ifundefined{#1@cntformat}%
   {\csname the#1\endcsname\quad}%    default
   {\csname #1@cntformat\endcsname}}% enable individual control
\newcommand\section@cntformat{}     % section level 
\makeatother

\lstset{
     extendedchars=true,
     literate=%
         {á}{{\'a}}1
         {í}{{\'i}}1
         {é}{{\'e}}1
         {è}{{\`e}}1
         {ý}{{\'y}}1
         {ú}{{\'u}}1
         {ó}{{\'o}}1
         {ě}{{\v{e}}}1
         {š}{{\v{s}}}1
         {č}{{\v{c}}}1
         {ř}{{\v{r}}}1
         {ž}{{\v{z}}}1
         {ď}{{\v{d}}}1
         {ť}{{\v{t}}}1
         {ň}{{\v{n}}}1
         {ů}{{\r{u}}}1
         {Á}{{\'A}}1
         {Í}{{\'I}}1
         {É}{{\'E}}1
         {Ý}{{\'Y}}1
         {Ú}{{\'U}}1
         {Ó}{{\'O}}1
         {Ě}{{\v{E}}}1
         {Š}{{\v{S}}}1
         {Č}{{\v{C}}}1
         {Ř}{{\v{R}}}1
         {Ž}{{\v{Z}}}1
         {Ď}{{\v{D}}}1
         {Ť}{{\v{T}}}1
         {Ň}{{\v{N}}}1
         {Ů}{{\r{U}}}1
}

\hypersetup{
  colorlinks=true,
  linkcolor=blue!50!red,
  urlcolor=green!70!black
}
\numberwithin{equation}{section}

\newcommand{\sol}[1]{
    %{\color{blue} #1}
    }
\renewcommand{\deg}[1]{\:^\circ\text{#1}}

\renewcommand{\familydefault}{\sfdefault}

\renewcommand*\DTstylecomment{\rmfamily}

\definecolor{codegray}{rgb}{0.0,0.0,0.7}
\lstset{float,
  commentstyle=\color{codegray},
  basicstyle=\scriptsize,
  columns=fullflexible,
  numbers=left,
  numberstyle=\tiny,
  numbersep=5pt,
  xleftmargin=5pt,
  frame=tb,
  captionpos=b,
  mathescape=true,
  tabsize=2,
  showtabs=true,
  escapeinside={(*@}{@*)}}


\newboolean{showcomments}
\setboolean{showcomments}{true}
\ifthenelse{\boolean{showcomments}}
{ \newcommand{\mynote}[3]{
    \fbox{\bfseries\sffamily\scriptsize#1}
    {\small$\blacktriangleright$\textsf{\emph{\color{#3}{#2}}}$\blacktriangleleft$}}}
{ \newcommand{\mynote}[3]{}}
\newcommand{\shrink}[1]{}
% One command per author:
\newcommand{\pf}[1]{\mynote{Pascal}{#1}{magenta}}
\newcommand{\cg}[1]{\mynote{Christian}{#1}{orange}}
\newcommand{\rd}[1]{\mynote{R\'emi}{#1}{purple}}
\newcommand{\answer}[1]{\mynote{Answer:}{#1}{blue}}

\makeatletter
    \def\blfootnote{\xdef\@thefnmark{}\@footnotetext}
\makeatother

%%%%%%%%%%%%%%%%%%%%%%%%%%%%%%%%

\quand{2023-2024}
\siglemat{Artificial Intelligence}
\sorte{Week}
\formation{Computer science department}
\titre{Assignment 4}

\usepackage{courier}

\begin{document}
\thispagestyle{plain}
\mettreletitre

\blfootnote{Questions ? victor.villin@unine.ch}

\section*{Subject}

Topics of this session:
\begin{enumerate}
\item Constrained Search.
\item Logical Inference, knowledge-based agents.
\item Formalise a problem as a constrained graph search problem.
\end{enumerate}
\textbf{This assignement is graded and must be submitted (individually) on Moodle before next week's class}. 

\smallskip
For each exercise, detail your reflexion steps:
\begin{itemize}
    \item We are mostly interested in your actual thinking process.
    \item Even if you are unable to solve an exercise, write out what were you reflexion steps.
    \item For each attempted exercise, a written feedback will be provided. 
\end{itemize}

\smallskip
Reference material: \href{https://github.com/yanshengjia/ml-road/blob/master/resources/Artificial%20Intelligence%20-%20A%20Modern%20Approach%20(3rd%20Edition).pdf}{Artificial Intelligence, A Modern Approach, Chapters 6, 7}.

For coding:
\begin{itemize}
    \item \href{https://noto.epfl.ch}{Noto} (Online Jupyter NoteBook).
    \item Any other python coding environment you prefer using.
\end{itemize}

\section*{Exercise 1}

Let's consider a train scheduling problem.
We must ensure that connections are all feasible. We assume that a connection between train A and B at station S is feasible if train A arrives at S exactly when B leaves S. 

Here are the required connections (this is simplified):
\begin{itemize}
    \item The Neuchâtel-Lausanne train must connect to the Lausanne-Geneva and Lausanne-Bern.
    \item The Lausanne-Geneva train must connect to the Geneva-Versoix train.
    \item The Bern-Lausanne train must connect to the Lausanne-Geneva.
\end{itemize}

Here are the possible times for departure/arrival for each trains:
\begin{itemize}
    \item Neuchâtel-Lausanne: (08:30/09:30) -- (09:30/10:30) -- (10:30/11:30)
    \item Lausanne-Geneva: (07:30/08:30) -- (08:30/09:30) -- (09:30/10:30) -- (10:30/11:30)
    \item Lausanne-Bern: (09:00/09:30) -- (10:00/11:30)
    \item Geneva-Versoix: (09:30/09:50) -- (10:00/10:20) -- (10:30/10:50)
    \item Bern-Lausanne: (07:00/08:30) -- (08:00/09:30) -- (09:00/10:30)
\end{itemize}

\textbf{Task:}
\begin{itemize}
    \item Define appropriate variables to formalise the problem.
    \item Find a valid solution to the problem by drawing a graph.
\end{itemize}

\section*{Exercise 2}

Here is an example of how to solve a problem through logical inference:

\smallskip
\textbf{Example:} 
\begin{itemize}
    \item If it rains, then the ground is wet.
    \item If the ground is wet, then the grass is slippery.
    \item The grass is not slippery.
\end{itemize}
\textbf{Question:}  Did it rain or not ?

\paragraph{Notations}
\begin{itemize}
    \item Rain: $R$.
    \item The ground is wet: $W$.
    \item The grass is slippery: $S$.
\end{itemize}

\paragraph{Knowledge base} Each of the following statements are true:
\begin{enumerate}
    \item If it rains, then the ground is wet: $R \implies W$.
    \item If the ground is wet, then the grass is slippery: $ W \implies S$.
    \item The grass is not slippery: $\neg S$.
\end{enumerate}

\paragraph{Deduction}
Using the rule of the contraposition, we have:
\begin{itemize}
    \item[] $ \neg S \implies \neg W$. \quad\quad (1.)
    \item[] $ \neg W \implies \neg R$. \quad\quad (2.)
    \item[] Thus, since $\neg S$ is true, \textbf{it did not rain}.
\end{itemize}

\bigskip
Following this example, answer the following questions:

\begin{enumerate}
    \item Considering the same problem as the one given in the example, let's assume additionally that the ground is wet if the grass is slippery, and that the grass is slippery. 
    
    \textbf{Quesiton:} Did it rain or not ?

    \sol{
        On top of knowing that the grass is slippery ($S$), we have a new rule: $S \implies W$.
        We thus know that the ground is wet as well: $W$.

        However, we cannot tell anything about the rain, as the grass being slippery or the ground being wet does not imply anything about the rain: $(S \land W) \centernot\implies R$. 
        We thus cannot say anything about the rain.
    }

    \item You are trying to catch a train, and you know for a fact that:
    \begin{itemize}
        \item If the train is late, you \emph{will} catch it.
        \item If the train is on time and you leave on time, you \emph{will} catch it.
        \item If you leave late, you \emph{will} miss the train.
        \item You left late.
    \end{itemize}
    \textbf{Question:} Can you infer whether the train was late or not?

    \sol{
        \paragraph{Notations}
        \begin{itemize}
            \item The train is on time: $T$.
            \item You left late: $L$.
            \item You catch the train: $C$.
        \end{itemize}

        \paragraph{Knowledge base}
        \begin{enumerate}
            \item If the train is late, you \emph{will} catch it: $\neg T \implies C$.
            \item If the train is on time and you leave on time, you \emph{will} catch it: $(T \land \neg L) \implies C$.
            \item If you leave late, you \emph{will} miss the train: $L \implies \neg C$.
            \item You left late: $L$.
        \end{enumerate}

        \paragraph{Deduction}
        We must infer whether the train was late or not.
        \begin{itemize}
            \item We left late, so that means that we missed the train (c): $\neg C$.
            \item By contraposition of (b): $\neg C \implies (\neg T \lor L)$. This rule cannot inform us about ($T$) since we are late ($L$).
            \item By contraposition of (a): $\neg C \implies T$.
            %\item Rule (b) does not provide anything additional.
            \item With the latter, we inferred that \textbf{the train was on time}.
        \end{itemize}
    }

    \iffalse
    \item We are wondering about the relationships between Alice, Bob, and Charlie. We have that:
    
    \begin{itemize}
        \item If Bob is friends with Charlie, then Charlie is friends with Alice.
        \item If Alice is friends with Bob, and Charlie is friends with Bob, then Alice is also friends with Charlie.
        \item If Alice is not friends with Bob, then Charlie is not friends with Alice.
        \item Bob and Charlie are friends.
    \end{itemize}
    \textbf{Question:} Prove whether Alice and Charlie are friends.
    
    \sol{
        \paragraph{Notations}
        \begin{itemize}
            \item Alice is friends with Bob: $A_B$.
            \item Alice is friends with Charlie: $A_C$.
            \item Bob is friends with Alice: $B_A$.
            \item Bob is friends with Charlie: $B_C$.
            \item Charlie is friends with Alice: $C_A$.
            \item Charlie is friends with Bob: $C_B$.
        \end{itemize}

        \paragraph{Knowledge base}
        \begin{enumerate}
            \item If Bob is friends with Charlie, then Charlie is friends with Alice: $B_C \implies C_A$.
            \item If Alice is friends with Bob, and Charlie is friends with Bob, then Alice is also friends with Charlie: $(A_B \land C_B) \implies A_C$.
            \item If Alice is not friends with Bob, then Charlie is not friends with Alice: $\neg A_B \implies \neg C_A$.
            \item Bob and Charlie are friends: $B_C \land C_B$.
        \end{enumerate}

        \paragraph{Deduction}
        We must infer the relationship between Alice and Charlie.
        \begin{itemize}
            \item Since Bob is friends with Charlie (d), we know that Charlie is friends with Alice (a): $C_A$.
            \item Knowing that $C_A$ is true, through the contraposition of (c), $C_A \implies A_B$, we know that Alice is friends with Bob.
            \item Finally, since Charlie (d) and Alice are friends with Bob, we know through (b) that $A_C$ must be also true.
            \item Thus, \textbf{Alice and Charlie are friends}: $C_A \land A_C$.
        \end{itemize}
    }
    \fi

    \item Your friend is complaining about cramps. You know that they are practicing \emph{only} one of the four possible sports: Swimming, Running, Basketball, or Cycling.
    You wish to guess which sport they are actually practicing. Let's assume that:
    \begin{itemize}
        \item \emph{Only} Swimming, Running, and Basketball \emph{can} cause cramps.
        \item Basketball and Cycling \emph{always} make you thirsty.
        \item Running \emph{cannot} make you dirty \emph{and} thirsty at the same time.
    \end{itemize}
    \textbf{Question:} Under which additional condition do you know for sure that your friend practices Basketball?
    \sol{

\paragraph{Given Conditions}
We define the following logical statements:
\begin{itemize}
    \item $S$ = Swimming.
    \item $R$ = Running.
    \item $B$ = Basketball.
    \item $C$ = Cycling.
    \item $X$ = The friend has cramps.
    \item $Y$ = The friend is thirsty.
    \item $Z$ = The friend is dirty.
\end{itemize}

\paragraph{Knowledge Base}
\begin{enumerate}
    \item The friend practices one sport: $S \lor R \lor B \lor C$
    \item The friend practices \emph{only} one sport: $\neg (S \land R) \land \neg (S \land B) \land \neg (S \land C) \land \neg (R \land B) \land \neg (R \land C) \land \neg (B \land C)$
    \item Cramps occur only if the sport is Swimming, Running, or Basketball: $X \implies (S \lor R \lor B)$.
    \item Basketball and Cycling always make you thirsty: $B \land C \implies Y$.
    \item Running cannot make you both thirsty and dirty: $(Y \land Z) \implies \neg R$.
    \item The friend has cramps: $X$.
\end{enumerate}

\paragraph{Deduction}
We want to find under what conditions we can deduce that our friend is practicing Basketball ($B$).
\begin{itemize}
    \item Since the friend has cramps (f), they must be doing Swimming, Running, or Basketball (c): $S \lor R \lor B$.
    \item Since we know that the friend practices only one sport (a \& b), we can rule out Cycling: $\neg C$.
    \item If we further observe that the friend is \textbf{thirsty and dirty} ($Y \land Z$), then Running is ruled out because Running cannot make someone both dirty and thirsty (e by contraposition): $\neg R$
    \item If we additionally observe that the sport \textbf{always makes them thirsty}, then Swimming must be ruled out, because Swimming does not always make people thirsty (d, $S \centernot\implies Y$) : $\neg S$
    \item The only remaining sport is Basketball.
\end{itemize}

\paragraph{Conclusion}
If we additionally observe that the friend is \textbf{Dirty} and \textbf{always Thirsty}, then we can infer that the friend is practicing Basketball ($B$). 
    } 

\end{enumerate}

\section*{\textbf{Project -- Part 2}}

\textbf{This whole section must be done in groups of \underline{2-3 people}.}
\smallskip

Reutilising what we did last week, we will add \textbf{constraints} and \textbf{logical rules} to our problem.

\subsection{Guided Project}

\paragraph{Dungeon Gridworld}

\begin{figure}[h]
    \centering
\begin{tikzpicture}[scale=1]

    % Draw outer walls (5x5 grid with walls around it)
    \draw[thick] (-1, -1) rectangle (6, 6);

    % Draw internal grid
    \foreach \x in {0,...,5} {
        \foreach \y in {0,...,5} {
            \draw (\x, \y) rectangle ++(1, 1);
        }
    }
    
    % Draw surrounding walls as filled gray cells
    \foreach \pos in {
        (-1,-1), (0,-1), (1,-1), (2,-1), (3,-1), (4,-1), (5,-1), (5,-1),
        (-1,2), (5,2),
        (-1,3), (5,3),
        (-1,4), (5,4),
        (-1,5), (5,5),
        (-1,5), (0,5), (1,5), (2,5), (3,5), (4,5), (5,5), (5,5),
        (-1, 1), (-1, 0), (5, 1), (5, 0),
        % walls
        (2, 3), (2, 2), (0, 2), (2, 0)
    } {
        \fill[black] \pos rectangle ++(1, 1);
    }

    % Start position
    \node at (0.5, 0.5) {\includegraphics[width=0.5cm]{happy-man.png}};

    % Holes
    \node at (0.5, 4.5) {\includegraphics[width=0.5cm]{hole.png}};
    \node at (3.5, 2.5) {\includegraphics[width=0.5cm]{hole.png}};


    % Stairs
    \node at (3.5, 4.5) {\includegraphics[width=0.5cm]{stairs.png}};


    % Add small coordinate labels at the bottom right
    \foreach \x in {0,...,5} {
        \foreach \y in {0,...,5} {
            \node[scale=0.5, anchor=south east] at (\x + 0.98, \y + 0.01) {\small (\x, \y)};
        }
    }

    % sensing
    %\draw[red, line width=1.5pt] (-0.99, -0.99) rectangle (1.99, 1.99);
    %\node[anchor=south, scale=0.75] at (0.01, 2) {\color{red} Sensing Range};

\end{tikzpicture}
\label{fig:env}
\caption{A simple dungeon room with holes. The agent starts at position (0, 0).}
\end{figure}

This time, we will set a more constrained observation space as well as some logic to our problem.

\textbf{New Assumptions:}
\begin{itemize}
    \item {\color{gray} The goal is to reach the stairs \textbf{as quickly as possible}.}
    \item {\color{gray} Falling into a hole causes the game to end immediately.}
    \item The agent \textbf{has partial visibility}, and only senses adjecent tiles:
    \begin{itemize}
        \item[-]  Whenever the agent is adjacent to a hole, it hears an \emph{Echo}.
        \item[-]  Whenever the agent hits a wall, it senses a \emph{Bump}.
        \item[-]  Whenever the agent is in the same row or column as the stairs, it observes \emph{Light}.
    \end{itemize}
    \item Each time the agent picks an action, it observes $\{e, b, l\}$, where $e=1$ if the agent hears an \emph{Echo} and 0 otherwise,  $b=1$ if the action resulted in no movements and 0 otherwise, etc.
    \item additionally, we have the following \textbf{rules}:
    \begin{itemize}
        \item[-]  Holes are \emph{always} on the same row or column as the stairs.
        \item[-]  The stairs are \emph{never} adjacent to a hole.
        \item[-]  If a tile is a wall, there \emph{cannot} be any agent, hole or stairs there.
    \end{itemize}
\end{itemize}

\textbf{Tasks:}
\begin{enumerate}
    \item Using logic notations, and by introducing appropriate variables, formalise under which conditions the agent sense an \emph{Echo}, a \emph{Bump} and \emph{Light}.
    \item Let's say the agent performed the actions [↑,↑,→,→,→]. What is the knowledge base of the agent at this point in time ? 
    
    \sol{
        \paragraph{Knowledge Base}
        \begin{enumerate}
        \item We bumped into a wall after the second action, thus, there is a wall at (0, 2). This also means that we walked the following path: [(0,0), (0,1), (1,1), (2,1), (3,1)].
        \item There were no light at (0,0), (0,1), (1,1), (2,1), thus, $y_S \centernot \in \{0,1,2\}$ and  $x_S \centernot \in \{0, 1\}$.
        \item There were light at (3,1), it could either mean that $x_S =3$ or that $y_S=1$. However, knowing that $y_S \centernot\in \{0, 1\}$, it implies that $x_S=3$.
        \item Upon reaching tile (3,1), we heard an echo. This means that there is a hole either in (3,2), (4,1) or (3,0). (2,1) is not an option since we walked on that tile. 
        (4,1) is not an option either since the hole must be on the same row or column than the stairs, but $x_S\neq 4$ and $y_S \neq 1$.
        \end{enumerate}
    }

    \item Let's further imagine that the agent keeps going and performs next [→, ↑, ↑,↑]. What is the knowledge base at this point in time? Enumerate all possibilities for tile (3,4): does it have stairs, a wall, a hole?
    \item Think about one constraint you could have in this problem. 
    %\item Implement a simple logical inference module (Belief State update logic) using the template code in \texttt{dungeon\_with\_constraints.ipynb}.
    %\item Implement a simple agent using that inference module (\texttt{your\_agent()} function in \\ \texttt{dungeon\_with\_constraints.ipynb}).
\end{enumerate}

\subsection{Personal Project}

Add some constraints and/or partial observability to your problem, then, \textbf{if applicable}:
\begin{enumerate}
\item Using logic notations, and by introducing appropriate variables, formalise the new constraints and/or formalise under which conditions observations are made. 
%\item Implement and try solving the problem with the added sensing and constraints in Python.
\end{enumerate}

\end{document}
